\documentclass[12pt]{ltjarticle}
\usepackage{enumitem}
\usepackage{caption}
\usepackage{graphicx}
\usepackage[hyphens]{url}
\usepackage{multirow}
\usepackage{float}
\usepackage{amsmath}
\usepackage{here}
\usepackage{array}
\usepackage{geometry}
\usepackage{microtype}
\usepackage{fontspec}
\usepackage{luatexja-fontspec} % 明示的に読み込む
\usepackage{luatexja}
\usepackage{titlesec}
\usepackage{subcaption}
\usepackage{listings}

% ========== listings の設定 ==========
\lstset{
  % フレーム (枠線) の指定:上 (t) と 下 (b) の線だけ
  frame=tb, 
  % 枠線とコードの間の余白
  framesep=5pt, 
  % 等幅フォントで表示
  basicstyle=\ttfamily\small, 
  % 行番号を左側に表示
  numbers=left,
  % 行番号の書式を小さく(tiny)する
  numberstyle=\tiny, 
  % 行番号とコードの間のスペース
  numbersep=5pt, 
  % 行番号の増分 (1行ごと)
  stepnumber=1, 
}

% \usepackage{array}
% \usepackage{booktabs}
% \geometry{margin=20mm}
\titleformat{\section}[block]
  {\normalfont\LARGE\bfseries}
  {第\thesection 章}{1em}{}

\setmainfont{ipaexm.ttf} % 日本語フォント(IPAex明朝)を指定 ※ XeLaTeX or LuaLaTeXが必要
\geometry{left=25mm, right=25mm, top=30mm, bottom=30mm} % 余白設定
\renewcommand{\baselinestretch}{1.2} % 行間を1.2倍に設定

\counterwithin{figure}{section} % 図番号を「1.1」のようにセクションと連動
\counterwithin{table}{section}  % 表番号も同様にセクションと連動

\makeatletter
% section 再定義(改ページなし・見た目調整)
\renewcommand{\section}{%
  \@startsection{section}{1}{\z@}%
  {-2.5ex \@plus -1ex \@minus -.2ex}%
  {1.5ex \@plus.2ex}%
  {\normalfont\LARGE\bfseries\raggedright}%
}

% subsection 再定義(見た目だけ調整,番号はそのまま)
\renewcommand{\subsection}{%
  \@startsection{subsection}{2}{\z@}%
  {-1.5ex \@plus -0.5ex \@minus -.2ex}%
  {1ex \@plus .2ex}%
  {\normalfont\large\bfseries\raggedright}%
}

% section 番号だけ「第n章」に変更
\renewcommand{\thesection}{第\arabic{section}章}

% subsection 番号は従来の n.m 形式を保つ(これが重要)
\renewcommand{\thesubsection}{\arabic{section}.\arabic{subsection}}
\makeatother

\makeatletter
% 表番号を「2.1」「2.2」のようにセクション番号と連動
\renewcommand{\thetable}{\arabic{section}.\arabic{table}}
% キャプションの前に「表」を表示
\renewcommand{\tablename}{表}
% 参照の際にも「表」と表示
\renewcommand{\refname}{\tablename}
\renewcommand{\thefigure}{\arabic{section}.\arabic{figure}}
\renewcommand{\figurename}{図}
\makeatother

% キャプションの書式設定
\captionsetup[figure]{labelfont={bf}, labelsep=space, font=small}  % キャプションを小さくし,「図」を太字に設定

% タイトルと著者の設定
\title{\huge 第5章 サイバーセキュリティ基礎実験2}
\author{\large 3年情報工学科 19番 瀧口大地}
  
\date{} % 日付を空にする

\begin{document}

  \begin{titlepage}
    \vspace*{\fill}
    \begin{center}
      {\Large 「好きなもの」のデータベースWebアプリケーション}\\
      \vspace{0.5\baselineskip}
      {\Huge 積みゲー管理アプリ}
    \end{center}

    \vspace{2cm}
    \begin{LARGE}
      \begin{center}
        3年 情報工学科 19番 瀧口大地
      \end{center}
    \end{LARGE}

    \begin{large}
    \vspace{1.5cm}
    \begin{flushleft}
      \normalsize
      提出期限: 2026年2月15日23:59\\
      提出日: 2026年2月15日23:59
    \end{flushleft}

    \vspace{1cm}
    \begin{flushleft}
       \\
       \\
       \\
       \\
       \\
       \\
       \\
       \\
       \\
    \end{flushleft}
    \end{large}

    \vspace*{\fill}
  \end{titlepage}

  \clearpage
  \setcounter{page}{1}
  \pagestyle{plain}
  
  \newpage
  \noindent
  % \setcounter{page}{1}
  \section{目的}
  今回私が「積みゲー管理アプリ」を作ろうと思った目的として,背景と解決手段を以下に示す.

  \subsection{背景}
  ここ最近,ゲームストアの革新的なセールや有料ゲームの期間限定無料配布により,私の
  PCに大量のゲームがインストールされた.量が量である故,購入して未プレイのゲーム,
  所謂「積みゲー」が増加してしまっている.複数のゲームを楽しんでいるとどうしても
  進行状況やプレイ時間などの進行度を忘れてしまう.これらの現象が頻発すると,せっかくの
  ゲームを最高に楽しむことが難しくなってくる.私はこれを強く問題視した.

  \subsection{解決手段}
  私は今上げた問題の解決手段として,シンプルなものを思いついた.記録である.
  忘れてしまったり薄れてしまう情報の対策として,記録と確認は絶大な効果がある.
  高専の方々に伝わりやすい例で言うと,githubのコミットで変更点や現状の説明を
  文章として保存する,ああいった記録を本問題の解決手段として用いる.
  システムとしてはPHPとCSVを用いてゲームごとの進行ログ(日記)を保存できる
  Webアプリを開発する.累計時間や現在のステータス,過去のプレイ履歴を振り返る
  機能を持たせることで,モチベーションの維持や効率的で高い質のゲームプレイを提供する.

  \newpage
  \noindent
  \section{Webインターフェースの設計}
  Webインターフェースについて,Webアプリの概要とUIやUXなどの細かい工夫に分けて述べる.
  
  \subsection{Webアプリの概要}
  本システムにアクセスするとまず図\ref{Webアプリ初期画面}のようになっている.

  \begin{figure}[H]
    \centering
    \includegraphics[width=0.99\textwidth]{datas/Webアプリ初期画面.png}
    \caption{Webアプリ初期画面}
    \label{Webアプリ初期画面}
  \end{figure}

  ADD\_NEW\_GAMEのTITLEにゲームのタイトル,LENGTHにゲームのクリアにかかるであろう時間
  を入力する.その後ENTERキーの押下またはREGISTERボタンで初期入力は完了である.
  図\ref{Webアプリ初期入力}のように入力しENTERキーを押すと図\ref{Webアプリ初期出力}
  のようにADD\_NEW\_GAMEの下にゲームが1つのブロックとして記録される.

  \begin{figure}[H]
    \centering
    \includegraphics[width=0.99\textwidth]{datas/Webアプリ初期入力.png}
    \caption{Webアプリ初期入力}
    \label{Webアプリ初期入力}
  \end{figure}

  \begin{figure}[H]
    \centering
    \includegraphics[width=0.99\textwidth]{datas/Webアプリ初期出力.png}
    \caption{Webアプリ初期出力}
    \label{Webアプリ初期出力}
  \end{figure}

  図\ref{Webアプリ初期出力}からわかるように,出力されたブロックにさらに入力がある.
  ゲームを進めて進捗があった時,ここに図\ref{Webアプリ追加入力}のようにSTATUS
  を「プレイ中」とし,LOG\_COMMENTに感想や進捗コメント,プレイ時間を入力する.
  UPDATEボタンを押すと,図\ref{Webアプリ追加出力}のようにSTATUSが変化し
  現状のプレイ時間が記録される.また,いつどんな体験をしたのかが一目でわかるように
  ゲームごとに履歴が保存できるようになっている.

  \begin{figure}[H]
    \centering
    \includegraphics[width=0.99\textwidth]{datas/Webアプリ追加入力.png}
    \caption{Webアプリ追加入力}
    \label{Webアプリ追加入力}
  \end{figure}

  \begin{figure}[H]
    \centering
    \includegraphics[width=0.99\textwidth]{datas/Webアプリ追加出力.png}
    \caption{Webアプリ追加出力}
    \label{Webアプリ追加出力}
  \end{figure}

  以上により,本Webアプリのメイン処理の説明は完了である.ただ,
  図\ref{Webアプリ初期画面}の画面右上にDATA\_PAGEボタン,DOWLOAD\_ALLボタンが
  あることがわかる.DOWLOAD\_ALLボタンはその名の通り現状のゲーム状況をまとめた
  データベースのCSVファイルを全てダウンロードする,というものである.
  DATA\_PAGEボタンを押下すると図\ref{Webアプリ別画面}のようにページが遷移し,
  各データベースのテーブルを閲覧できるようになっている.図内の「progress\_1767609780.csv」
  は,先ほど登録したシルクソングというゲームの進捗を記録したテーブルを示している.
  1767609780はgames.csvで定義されたIDであり,詳しくはデータベース設計で後述するが
  ゲームごとにそれぞれ進捗をまとめられるようにしている.

  \begin{figure}[H]
    \centering
    \includegraphics[width=0.99\textwidth]{datas/Webアプリ別画面.png}
    \caption{Webアプリ別画面}
    \label{Webアプリ別画面}
  \end{figure}
  
  最後に本Webアプリの使用感を示すために他のゲームも追加した画面を載せておく.

  \begin{figure}[H]
    \centering
    \includegraphics[width=0.99\textwidth]{datas/Webアプリ使用感.png}
    \caption{Webアプリ使用感}
    \label{Webアプリ使用感}
  \end{figure}

  \subsection{UIやUXの細かい工夫}

  \newpage
  \noindent
  \section{データベースの設計}
  \newpage
  \noindent
  \section{実装スケジュール}
  \newpage
  \noindent
  \section{まとめ}

  \newpage
  \begin{huge}
    参考文献\\\\
  \end{huge}
  % \noindent[]ページ名,\url{},2025年12/25参照.\\

  \newpage
  \begin{huge}
    感想\\\
  \end{huge}
  % \noindent[]ページ名,\url{},2025年12/25参照.\\
\end{document}
