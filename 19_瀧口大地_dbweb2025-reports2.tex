\documentclass[12pt]{ltjarticle}
\usepackage{enumitem}
\usepackage{caption}
\usepackage{graphicx}
\usepackage[hyphens]{url}
\usepackage{multirow}
\usepackage{float}
\usepackage{amsmath}
\usepackage{here}
\usepackage{array}
\usepackage{geometry}
\usepackage{microtype}
\usepackage{fontspec}
\usepackage{luatexja-fontspec} % 明示的に読み込む
\usepackage{luatexja}
\usepackage{titlesec}
\usepackage{subcaption}
\usepackage{listings}
\usepackage[table]{xcolor}

% ========== listings の設定 ==========
\lstset{
  % フレーム (枠線) の指定:上 (t) と 下 (b) の線だけ
  frame=tb, 
  % 枠線とコードの間の余白
  framesep=5pt, 
  % 等幅フォントで表示
  basicstyle=\ttfamily\small, 
  % 行番号を左側に表示
  numbers=left,
  % 行番号の書式を小さく(tiny)する
  numberstyle=\tiny, 
  % 行番号とコードの間のスペース
  numbersep=5pt, 
  % 行番号の増分 (1行ごと)
  stepnumber=1, 
}

% \usepackage{array}
% \usepackage{booktabs}
% \geometry{margin=20mm}
\titleformat{\section}[block]
  {\normalfont\LARGE\bfseries}
  {第\thesection 章}{1em}{}

\setmainfont{ipaexm.ttf} % 日本語フォント(IPAex明朝)を指定 ※ XeLaTeX or LuaLaTeXが必要
\geometry{left=25mm, right=25mm, top=30mm, bottom=30mm} % 余白設定
\renewcommand{\baselinestretch}{1.2} % 行間を1.2倍に設定

\counterwithin{figure}{section} % 図番号を「1.1」のようにセクションと連動
\counterwithin{table}{section}  % 表番号も同様にセクションと連動

\makeatletter
% section 再定義(改ページなし・見た目調整)
\renewcommand{\section}{%
  \@startsection{section}{1}{\z@}%
  {-2.5ex \@plus -1ex \@minus -.2ex}%
  {1.5ex \@plus.2ex}%
  {\normalfont\LARGE\bfseries\raggedright}%
}

% subsection 再定義(見た目だけ調整,番号はそのまま)
\renewcommand{\subsection}{%
  \@startsection{subsection}{2}{\z@}%
  {-1.5ex \@plus -0.5ex \@minus -.2ex}%
  {1ex \@plus .2ex}%
  {\normalfont\large\bfseries\raggedright}%
}

% section 番号だけ「第n章」に変更
\renewcommand{\thesection}{第\arabic{section}章}

% subsection 番号は従来の n.m 形式を保つ(これが重要)
\renewcommand{\thesubsection}{\arabic{section}.\arabic{subsection}}
\makeatother

\makeatletter
% 表番号を「2.1」「2.2」のようにセクション番号と連動
\renewcommand{\thetable}{\arabic{section}.\arabic{table}}
% キャプションの前に「表」を表示
\renewcommand{\tablename}{表}
% 参照の際にも「表」と表示
\renewcommand{\refname}{\tablename}
\renewcommand{\thefigure}{\arabic{section}.\arabic{figure}}
\renewcommand{\figurename}{図}
\makeatother

% キャプションの書式設定
\captionsetup[figure]{labelfont={bf}, labelsep=space, font=small}  % キャプションを小さくし,「図」を太字に設定

% タイトルと著者の設定
\title{\huge 第5章 サイバーセキュリティ基礎実験2}
\author{\large 3年情報工学科 19番 瀧口大地}
  
\date{} % 日付を空にする

\begin{document}

  \begin{titlepage}
    \vspace*{\fill}
    \begin{center}
      {\Large 「好きなもの」のデータベースWebアプリケーション}\\
      \vspace{0.5\baselineskip}
      {\Huge 積みゲー管理アプリ}
    \end{center}

    \vspace{2cm}
    \begin{LARGE}
      \begin{center}
        3年 情報工学科 19番 瀧口大地
      \end{center}
    \end{LARGE}

    \begin{large}
    \vspace{1.5cm}
    \begin{flushleft}
      \normalsize
      提出期限: 2026年2月15日23:59\\
      提出日: 2026年2月15日23:59
    \end{flushleft}

    \vspace{1cm}
    \begin{flushleft}
       \\
       \\
       \\
       \\
       \\
       \\
       \\
       \\
       \\
    \end{flushleft}
    \end{large}

    \vspace*{\fill}
  \end{titlepage}

  \clearpage
  \setcounter{page}{1}
  \pagestyle{plain}
  
  \newpage
  \noindent
  % \setcounter{page}{1}
  \section{目的}
   今回私が「積みゲー管理アプリ」を作ろうと思った目的として,背景と解決手段を以下に示す.

  \subsection{背景[1]}
   ここ最近,ゲームストアの革新的なセールや有料ゲームの期間限定無料配布により,私の
  PCに大量のゲームがインストールされた.量が量である故,購入して未プレイのゲーム,
  所謂「積みゲー」が増加してしまっている.複数のゲームを楽しんでいるとどうしても
  進行状況やプレイ時間などの進行度を忘れてしまう.これらの現象が頻発すると,せっかくの
  ゲームを最高に楽しむことが難しくなってくる.私はこれを強く問題視した.

  \subsection{解決手段}
   私は今上げた問題の解決手段として,シンプルなものを思いついた.記録である.
  忘れてしまったり薄れてしまう情報の対策として,記録と確認は絶大な効果がある.
  高専の方々に伝わりやすい例で言うと,githubのコミットで変更点や現状の説明を
  文章として保存する,ああいった記録を本問題の解決手段として用いる.
  システムとしてはPHPとCSVを用いてゲームごとの進行ログ(日記)を保存できる
  Webアプリを開発する.累計時間や現在のステータス,過去のプレイ履歴を振り返る
  機能を持たせることで,モチベーションの維持や効率的で高い質のゲームプレイを提供する.

  \newpage
  \noindent
  \section{Webインターフェースの設計}
   Webインターフェースについて,Webアプリの概要とUIやUXなどの細かい工夫に分けて述べる.
  
  \subsection{Webアプリの概要}
   本システムにアクセスするとまず図\ref{Webアプリ初期画面}のようになっている.

  \begin{figure}[H]
    \centering
    \includegraphics[width=0.99\textwidth]{datas/Webアプリ初期画面.png}
    \caption{Webアプリ初期画面}
    \label{Webアプリ初期画面}
  \end{figure}

  ADD\_NEW\_GAMEのTITLEにゲームのタイトル,LENGTHにゲームのクリアにかかるであろう時間
  を入力する.その後ENTERキーの押下またはREGISTERボタンで初期入力は完了である.
  図\ref{Webアプリ初期入力}のように入力しENTERキーを押すと図\ref{Webアプリ初期出力}
  のようにADD\_NEW\_GAMEの下にゲームが1つのブロックとして記録される.

  \begin{figure}[H]
    \centering
    \includegraphics[width=0.99\textwidth]{datas/Webアプリ初期入力.png}
    \caption{Webアプリ初期入力}
    \label{Webアプリ初期入力}
  \end{figure}

  \begin{figure}[H]
    \centering
    \includegraphics[width=0.99\textwidth]{datas/Webアプリ初期出力.png}
    \caption{Webアプリ初期出力}
    \label{Webアプリ初期出力}
  \end{figure}

  図\ref{Webアプリ初期出力}からわかるように,出力されたブロックにさらに入力がある.
  ゲームを進めて進捗があった時,ここに図\ref{Webアプリ追加入力}のようにSTATUS
  を「プレイ中」とし,LOG\_COMMENTに感想や進捗コメント,プレイ時間を入力する.
  UPDATEボタンを押すと,図\ref{Webアプリ追加出力}のようにSTATUSが変化し
  現状のプレイ時間が記録される.また,いつどんな体験をしたのかが一目でわかるように
  ゲームごとに履歴が保存できるようになっている.

  \begin{figure}[H]
    \centering
    \includegraphics[width=0.99\textwidth]{datas/Webアプリ追加入力.png}
    \caption{Webアプリ追加入力}
    \label{Webアプリ追加入力}
  \end{figure}

  \begin{figure}[H]
    \centering
    \includegraphics[width=0.99\textwidth]{datas/Webアプリ追加出力.png}
    \caption{Webアプリ追加出力}
    \label{Webアプリ追加出力}
  \end{figure}

  以上により,本Webアプリのメイン処理の説明は完了である.ただ,
  図\ref{Webアプリ初期画面}の画面右上にDATA\_PAGEボタン,DOWLOAD\_ALLボタンが
  あることがわかる.DOWLOAD\_ALLボタンはその名の通り現状のゲーム状況をまとめた
  データベースのCSVファイルを全てダウンロードする,というものである.
  DATA\_PAGEボタンを押下すると図\ref{Webアプリ別画面}のようにページが遷移し,
  各データベースのテーブルを閲覧できるようになっている.図内の「progress\_1767609780.csv」
  は,先ほど登録したシルクソングというゲームの進捗を記録したテーブルを示している.
  1767609780はgames.csvで定義されたIDであり,詳しくはデータベース設計で後述するが
  ゲームごとにそれぞれ進捗をまとめられるようにしている.

  \begin{figure}[H]
    \centering
    \includegraphics[width=0.99\textwidth]{datas/Webアプリ別画面.png}
    \caption{Webアプリ別画面}
    \label{Webアプリ別画面}
  \end{figure}
  
  最後に本Webアプリの使用感を示すために他のゲームも追加した画面を載せておく.

  \begin{figure}[H]
    \centering
    \includegraphics[width=0.99\textwidth]{datas/Webアプリ使用感.png}
    \caption{Webアプリ使用感}
    \label{Webアプリ使用感}
  \end{figure}

  \subsection{UIやUXの細かい工夫[2][3][4]}
   これまでに添付した図の通り,本Webアプリは黒を基調としネオンカラーを用いている.
  これは本システムが他でもないゲーム管理アプリであり,用途にマッチした没入感のある
  デザインを構築するためである.また,記録するゲームごとにブロック分けをし,
  ブロック内で処理を完結させることでWebページ内をコンパクトにしている.
  また,昨今のアプリでは当たり前ではあるが,削除処理を赤色にしたり,ボタンの
  ホバー時の色をわかりやすく変えたりなどもしている.

  本Webアプリは単なるリストとしてシステム化することもできたが,ゲームごとの日記
  を紐付けたことで,自分がどのようにそれぞれのゲームに向き合ったか,という
  ストーリー性や懸けた熱に重点を置いた石器となっている.また,プレイ時間の勘違いや
  入力ミスを防ぐために時間の更新に下限を設けている(プレイ時間が15時間なら14以下は
  入力できないようになっている).また,一括ダウンロード機能や現在のデータベースを
  閲覧できるページの実装によりデータの所有権をユーザーに与えられるようにしている.

  \newpage
  \noindent
  \section{データベースの設計}
   本WebアプリのデータベースはWebアプリの概要で軽く触れたが2+n(n:ゲーム数)個
  のテーブルで成り立たせる.本章から用いるデータは図\ref{Webアプリ使用感}の状態のものである.

  \begin{figure}[H]
    \centering
    \includegraphics[width=0.99\textwidth]{datas/games.png}
    \caption{gamesテーブル}
    \label{games}
  \end{figure}

  図\ref{games}に示したgamesテーブルはシステム内ではgames.csvとして保存される,
  ゲーム本体の登録情報を管理する表である.項目(カラム)は左から順にID,タイトル,
  クリア想定時間,現在の状態IDとなっている.ゲームの種類が増えると行数が増えていく.

  \begin{figure}[H]
    \centering
    \includegraphics[width=0.99\textwidth]{datas/status_master.png}
    \caption{status\_masterテーブル}
    \label{status_master}
  \end{figure}

  図\ref{status_master}に示したstatus\_masterテーブルはシステム内でstatus\_
  master.csvとして保存される,gamesテーブルの項目(カラム)かつアプリ上で選択肢となる
  状態を定義する値をまとめている表である.項目(カラム)は左から順に状態ID,
  状態名となっている.

  \begin{figure}[H]
    \centering
    \includegraphics[width=0.99\textwidth]{datas/progress_silk.png}
    \caption{progress\_idテーブル}
    \label{progress_id}
  \end{figure}

  図\ref{progress_id}に示したprogress\_idテーブルはシステム内でprogress\_id
  (ここのidは各ゲームのもの)で保存される,ゲームごとの個別ログをまとめる表である.
  項目(カラム)は左から順にログID,記録日,その時の状態ID,コメント,累計時間と
  なっている.

  以上のようなテーブルが必要であると私は考えた.
  全ての進捗ログを1つのファイルに保存するのではなく,ゲームごとにテーブルを作成
  することで,ゲームの履歴を表示する際の読み込みが最適化され動作が高速に,データを
  簡潔にすることができる.またDBを読み込んで表示,変更があればDBの方を変更,という
  処理にすることでphp側で処理を行ってもcsvファイル側で処理を行っても
  常に整合性が取れるようにできる.

  \newpage
  \noindent
  \section{実装スケジュール[5]}

   本システムの開発における17時間の詳細ガントチャートを表\ref{unko}に示す.
  各マスは1時間を表し,色塗りはその時間に作業を行うことを示す.
  表のタスクIDに対する詳しい説明をさらに下にまとめる.\\

  \begin{table}[h]
    \centering
    \caption{タスクIDと時間(h)のガントチャート}
    \small % 文字サイズを標準的にして読みやすく
    \setlength{\tabcolsep}{0pt} % 余白を一旦ゼロにしてから
    \begin{tabular}{|l| *{17}{>{\centering\arraybackslash}p{0.7cm}|} } % 各マスの幅を0.7cmで固定
    \hline
    \textbf{タスクID / 時間} & 01 & 02 & 03 & 04 & 05 & 06 & 07 & 08 & 09 & 10 & 11 & 12 & 13 & 14 & 15 & 16 & 17 \\ \hline
    1 & \cellcolor[gray]{0.5} & \cellcolor[gray]{0.5} & \cellcolor[gray]{0.5} & & & & & & & & & & & & & & \\ \hline
    2 & & & & \cellcolor[gray]{0.5} & \cellcolor[gray]{0.5} & \cellcolor[gray]{0.5} & \cellcolor[gray]{0.5} & \cellcolor[gray]{0.5} & & & & & & & & & \\ \hline
    3 & & & & & & & & & \cellcolor[gray]{0.5} & \cellcolor[gray]{0.5} & \cellcolor[gray]{0.5} & \cellcolor[gray]{0.5} & & & & & \\ \hline
    4 & & & & & & & & & & & & & \cellcolor[gray]{0.5} & \cellcolor[gray]{0.5} & \cellcolor[gray]{0.5} & \cellcolor[gray]{0.5} & \cellcolor[gray]{0.5} \\ \hline
    \end{tabular}
    \label{unko}
  \end{table}
  
  \begin{description}[labelwidth=1.5em,labelsep=0.5em,leftmargin=!,style=nextline]
    \item[1:要件定義・DB設計]積みゲー管理に必要な項目を整理し,ステータスマスタの導入とゲーム別ファイル分割によるDB構造を決定する.
    \item[2:基本機能の実装]PHPによるCSVの読み書き,新規登録機能,および日記形式の追記機能を実装する.
    \item[3:データ構造の改善]1つのCSVに全ログを保存する形式から,ゲームIDごとの個別CSVファイルへ垂直分割する処理へリファクタリングを行う.
    \item[4:UI/UX調整・テスト]CSSによるレスポンシブデザインの適用,時間のバリデーションチェック,およびデータの削除確認ダイアログ等のテストを実施する.
  \end{description}

  \newpage
  \noindent
  \section{まとめ}
   本レポートでは,購入したまま未プレイとなっているいわゆる「積みゲー」を管理することを目的とした,Webアプリケーションの設計および実装について述べた.ゲームのタイトルや想定クリア時間,プレイ状況,実際のプレイ時間,感想などを記録・管理できる仕組みを構築することで,ゲームの進捗を可視化し,積みゲーの解消を支援することを目指した.
  本システムは,PHPを用いてサーバサイド処理を実装し,データベースの代替としてCSVファイルを用いる構成とした.ゲーム一覧を管理するCSVと,各ゲームごとの進捗ログを管理するCSVを分離することで,データ構造を簡潔に保ちつつ,処理の見通しを良くする設計とした.また,ステータスをマスタデータとして管理することで,表示や更新処理の柔軟性を高めている.
  Webインターフェースにおいては,ゲームの追加,進捗の更新,ログの削除,ゲームデータの削除といった基本的な操作を直感的に行えるよう設計した.特に,累計プレイ時間が過去の値より小さくならないよう制限を設けるなど,入力ミスを防ぐ工夫を行った.さらに,すべてのCSVデータを一括でダウンロードできる機能を実装し,データの保存性や管理のしやすさにも配慮した.
  本課題を通して,PHPによるフォーム処理やファイル入出力,CSVデータの扱いについての理解を深めることができた.また,単に機能を実装するだけでなく,利用者の操作性や実用性を意識した設計の重要性を学んだ.今回作成したシステムは簡易的なものであるが,今後はデータベースの導入や機能拡張を行うことで,より実用的な管理システムへ発展させることが可能であると考えられる.
  \newpage
  \begin{huge}
    参考文献\\\\
  \end{huge}
  % \noindent[]ページ名,\url{},2025年01/05参照.\\
  \noindent[1]積みゲーが増える理由を考えてみた,\url{https://note.com/phenol_adam/n/n44fdefd7ac4b},2026年01/05参照.\\
  \noindent[2]究極の『サイバーパンク2077』が今ここに,\url{https://www.cyberpunk.net/jp/ja/},2025年01/05参照.\\
  \noindent[3]UIとは!?今さら聞けない初心者がしっておくべきポイントをわかりやすく解説,\url{https://sales.fromation.co.jp/archives/10000063657},2025年01/05参照.\\
  \noindent[4]UXとは?UIとの違いやデザインの5要素 改善事例まで解説,\url{https://marketing.crexgroup.com/strategy-analysis/what-is-ux/},2025年01/05参照.\\
  \noindent[5]ガントチャートの基本と実践ステップ: WBS との違いも解説,\url{https://asana.com/ja/resources/gantt-chart-basics},2025年01/05参照.\\

  \newpage
  \begin{huge}
    感想\\
  \end{huge}
   作業が終わってから振り返ってみると,DBの扱いには手を焼いたが他はいつも行っていることと
  変わらないためスムーズに進行で来たな,と思った.何事も慣れが肝心であり,
  今後発表が課題として多くなっていくのもそういうことなんだろうなあと感じた.
  % \noindent[]ページ名,\url{},2025年12/25参照.\\
\end{document}
